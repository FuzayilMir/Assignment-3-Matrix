\documentclass[a4paper,12pt]{article}
\usepackage{graphicx}
\usepackage{amsmath}
\usepackage{booktabs}
\title{Assignment-3\\ Latex Report}
\author{Fuzayil Bin Afzal Mir}
\date{15/01/2021}
\begin{document}
	\maketitle
	
	
	\newpage
	\begin{itemize}
	    \item \Large\textbf{Exercise 2.101}
	\end{itemize}
	\section{Find the inverse and QR decomposition of this matrix
		\begin{pmatrix}
		 2 & -3\\
		 -1 & 2\\
    	\end{pmatrix}}\\$$
    	
\subsection{Solution} \\
\subsubsection{Inverse}
	Let the given marix be,\\
	
	A=\begin{pmatrix}
		 2 & -3\\
		 -1 & 2\\
    	\end{pmatrix}\\$$$$\\
    	To check whether the inverse of a matrix exists or not, if determinant of a matrix is non zero inverse exists, otherwise it dose not exist.\\
    	Now,
    	$$|A|= (2)*(2) - (-1)*(-3)\\
    	
    	$$|A|=4-3\\
    	
    	$$|A|=1\\
    	
    	Since adjoint of A is gretaer than 0 i.e non zero,therefore inverse of A exists.\\
    	Now,\\
    	we know,\\
    	
    	{A^{-1}}=\dfrac{1}{|A|}Adj(A)\\$$$$\\
    	
    	  {A^{-1}} = \dfrac{1}{1} \begin{pmatrix}
		 2 & 3\\
		 1 & 2\\
    	\end{pmatrix}\\$$$$\\
    	Hence,\\
    	 
    	 A^{-1} = \begin{pmatrix}
		 2 & 3\\
		 1 & 2\\
    	\end{pmatrix}\\
    	
    	
    	
    	
    	\subsubsection{QR Decomposition}
    	IN QR decomposition o a matrix , matrix is decomposed into the upper triangular matrix(R) and an orthogonal matrix(Q).\\
    	
     \\Since A=QR\\
   where,\\
   Q is an orthogonal matrix(i.e Q^T. Q = I)\\
   and\\
   
   R is an upper triangle matrix\\
    
    So,By using Gram-Schmidt method of decomposition let its column vectors be,\\ 
    
    
    v_1=\begin{pmatrix}
		 2\\
		 -1\\
    	\end{pmatrix} \hspace{2cm}{and} \hspace{2cm}{v_2=\begin{pmatrix}
		 -3\\
		  2\\
    	\end{pmatrix}}\\
	
	So,\\
	here\\
		
		Now, we know A=QR\\
		where,\\
		Q=(q1,q2)
		\\
		where ,\\
		q1 and q2 are column matrices.\\
		Now to determine q1 and q2\\
		we know\\
		
		q1=\dfrac{a}{length of a}\\
		
		therefore,\\
		
	q1 =\begin{gather}
	    \frac{1}{\sqrt{2^2+(-1)^2}}
	\begin{pmatrix}
		 2\\
		 -1\\
	\end{pmatrix}
		\end{gather}\\
		Here,\\
		\dfrac{1}{\sqrt{2^2+(-1)^2}} \\
		
		\\
		
		is the length of \begin{pmatrix}
		 2\\
		 -1\\
	\end{pmatrix}\\

	
		q1 = \begin{gather}
	    \frac{1}{\sqrt{4+1}}
	\begin{pmatrix}
		 2\\
		 -1\\
	\end{pmatrix}
	\end{gather}
	\\
	q1 = \begin{gather}
	    \frac{1}{\sqrt{5}}
	\begin{pmatrix}
		 2\\
		 -1\\
	\end{pmatrix}
		\end{gather}\\
	
		q1 = \begin{gather}
	\begin{pmatrix}
		   \frac{2}{\sqrt{5}}\\
		   \\
		     \frac{-1}{\sqrt{5}}
	\end{pmatrix}
		\end{gather}\\
	
	Now,\\
	
		
		q2=\dfrac{q2'}{length of q2'}\\

		Now,\\
		$$q2'=  v2-(v2.q1)q1\\

   
	$$q2' =\begin{pmatrix}
		 -3\\
		 2\\
    	\end{pmatrix}-\begin{pmatrix}
    
	\begin{pmatrix}
		 -3\\
		 2\\
	\end{pmatrix}. & 	
	\begin{pmatrix}
		 \frac{2}{\sqrt{5}}\\
		 \frac{-1}{\sqrt{5}}
	\end{pmatrix} 
	\end{pmatrix}\begin{pmatrix}
		 \frac{2}{\sqrt{5}}\\
		 \frac{-1}{\sqrt{5}}
	\end{pmatrix}\\$$$$\\
	
		$$q2' =\begin{pmatrix}
		 -3\\
		 2\\
    	\end{pmatrix}-\begin{pmatrix}
		 \frac{-8}{\sqrt{5}}
	\end{pmatrix}\begin{pmatrix}
		 \frac{2}{\sqrt{5}}\\
		  \frac{-1}{\sqrt{5}}
	\end{pmatrix}\\$$$$\\
	
	$$q2' =\begin{pmatrix}
		 -3\\
		 2\\
    	\end{pmatrix}-\begin{pmatrix}
		 \frac{-16}{5}\\
		 \\
		 \frac{8}{5}\\
	\end{pmatrix}\\$$$$\\
	
	
	
	$$q2' =\begin{pmatrix}
		 \frac{1}{5}\\
		 \\
		 \frac{2}{5}\\
	\end{pmatrix}\\$$$$\\
	
	
	Since,\\
		q2=\dfrac{q2'}{length of q2'}\\
	$$$$\\
	
	Therefore,\\
	
		$$q2=
	    \frac{1}{length of q2'}
	\begin{pmatrix}
		 \dfrac{1}{5}\\
		 \\
		 \dfrac{2}{5}\\
	\end{pmatrix}\\
	
	
		Here,\\
		
	length of q2'=    $\sqrt{(2/5)^2+(1/5)^2}\\
		
		length of q2'= $\sqrt{(4/25)+(1/25)}\\
		
		length of q2'= $$\sqrt{(5/25)}\\
		
		length of q2'= \dfrac{1}{\sqrt{5}}
		
		
		Hence,Using\\
		q2=\dfrac{q2'}{length of q2'}\\
	$$$$\\
	we get,\\
		
		q2=\dfrac{1}{\sqrt{5}}\begin{pmatrix}
		 \dfrac{1}{5}\\
		 \\
		 \dfrac{2}{5}\\
	\end{pmatrix}\\
		
		
		
		So,
		
			q2=\begin{pmatrix}
		 \dfrac{1}{\sqrt{5}}\\
		 \\
		 \dfrac{2}{\sqrt{5}}\\
	\end{pmatrix}\\
		
		
		Now,\\
		Q can be obtained by combining 
		\\q1
		
		and
		\\q2
		
		Therefore,
		
	Q=\begin{pmatrix}
		\frac{2}{\sqrt{5}} &
		
		 \frac{1}{\sqrt{5}}\\
		  \frac{-1}{\sqrt{5}} & 
		 \frac{2}{\sqrt{5}}\\
	\end{pmatrix}
		\end{gather}
		\\
		
		Since we know\\
		
		A=QR\\
		
		Pre-Multiplying both sides by Q^T,\\
		
		we get,\\
		
		Q^TA=Q^TQR\\
		
		But Q^TQ=I \hspace{4cm}\textbf{[I means identity matrix]}\\
		
		So,\\
		Q^TA=IR\\
		Q^TA=R   \\
		
		
		Or,\\
		
		R=Q^TA\\
		
		\textbf{Hence,}\\
		R=  \begin{pmatrix}
		\frac{2}{\sqrt{5}} &
		 \frac{-1}{\sqrt{5}}\\
		  \frac{1}{\sqrt{5}} & 
		   \frac{2}{\sqrt{5}}\\
	\end{pmatrix}\begin{pmatrix}
		 2 & -3\\
		 -1 & 2
    	\end{pmatrix}}
		\end{gather} \\
		
		\\
		\textbf{[By  matrix multiplication]}\\
		
		R=  \begin{pmatrix}
		\frac{4}{\sqrt{5}}+\frac{1}{\sqrt{5}} &
		 \frac{-6}{\sqrt{5}}-\frac{2}{\sqrt{5}}\\
		  \frac{2}{\sqrt{5}}- \frac{2}{\sqrt{5}}& 
		   \frac{-3}{\sqrt{5}}\frac{4}{\sqrt{5}}\\
	\end{pmatrix}\begin{pmatrix}
		 2 & -3\\
		 -1 & 2
    	\end{pmatrix}}
		\end{gather}\\
		
		\\
			R=  \begin{pmatrix}
		\frac{5}{\sqrt{5}} &
		 \frac{-8}{\sqrt{5}}\\
		 \\
		  0 & 
		   \frac{1}{\sqrt{5}}\\
	\end{pmatrix}
		\end{gather} 
		\\
		
		Also, from above we got\\
			
	{Q=\begin{pmatrix}
		\frac{2}{\sqrt{5}} &
		
		 \frac{1}{\sqrt{5}}\\
		 \\
		  \frac{-1}{\sqrt{5}} & 
		 \frac{2}{\sqrt{5}}\\
	\end{pmatrix}
		\end{gather}
		\\
	
	
		
		\newpage
    \begin{itemize}
    \item \Large\textbf{Exercise 2.102}
	\end{itemize}
	\section{Find the inverse and QR decomposition of this matrix
		\begin{pmatrix}
		 2 & 1\\
		 4 & 2\\
    	\end{pmatrix}}\\$$
    	
\subsection{Solution} \\
\subsubsection{Inverse}
	Let the given matrix be,\\
	
	A=\begin{pmatrix}
		 2 & 1\\
		 4 & 2\\
    	\end{pmatrix}\\$$$$\\
    	
    	we know,\\
    	{A^{-1}}=\dfrac{1}{|A|}Adj(A)\\$$$$\\
    	To check whether the inverse of a matrix exists or not, if determinant of a matrix is non zero inverse exists, otherwise it does not exist.\\
    	Now,
    	$$|A|= (2)*(2) - (4)*(1)\\
    	
    	$$|A|=4-4\\
    	
    	$$|A|=0\\
    	
    	Therefore,inverse of the given matrix A does not exist.\\
    	
    	
    	
    	
    	\subsubsection{QR decomposition}
    	
    	
     \\Since A=QR\\
   where,\\
   Q is an orthogonal matrix(i.e Q^T. Q = I)\\
   and\\
   
   R is an upper triangle matrix\\
    So, By using Gram-Schmidt method of decomposition let its column vectors be,\\ 
    v_1=\begin{pmatrix}
		 2\\
		 4\\
    	\end{pmatrix}\\
	and \\ v_2=\begin{pmatrix}
		 1\\
		  2\\
    	\end{pmatrix}\\
	
	Therefore,\\
	
	So,\\
	here\\
	\\$$\\

		Now, we know A=QR\\
		where,\\
		Q=(q1,q2)
		\\
		where ,\\
		q1 and q2 are column matrices.\\
		Now to determine q1 and q2\\
		we know\\
		
		q1=\dfrac{a}{length of a}\\
		
		therefore,\\
		

	q1 =\begin{gather}
	    \frac{1}{\sqrt{2^2+4^2}}
	\begin{pmatrix}
		 2\\
		 4\\
	\end{pmatrix}
		\end{gather}\\
		Here,\\
		\dfrac{1}{\sqrt{2^2+4^2}} \\
		
		\\
		
		is the length of \begin{pmatrix}
		 2\\
		 4\\
	\end{pmatrix}\\

	
		q1 = \begin{gather}
	    \frac{1}{\sqrt{20}}
	\begin{pmatrix}
		 2\\
		 4\\
	\end{pmatrix}
	\end{gather}
	\\

	
		q1 = \begin{gather}
	\begin{pmatrix}
		   \frac{2}{\sqrt{20}}\\
		   \\
		     \frac{4}{\sqrt{20}}
	\end{pmatrix}
		\end{gather}\\
	
	








	Now,\\
	
		
		q2=\dfrac{q2'}{length of q2'}\\

		Now,\\
		$$q2'=  v2-(v2.q1)q1\\

   
	$$q2' =\begin{pmatrix}
		 1\\
		 2\\
    	\end{pmatrix}-\begin{pmatrix}
    
	\begin{pmatrix}
		 1\\
		 2\\
	\end{pmatrix} .& 	
	\begin{pmatrix}
		 \frac{2}{\sqrt{20}}\\
		 \frac{4}{\sqrt{20}}
	\end{pmatrix} 
	\end{pmatrix}\begin{pmatrix}
		 \frac{2}{\sqrt{20}}\\
		 \frac{4}{\sqrt{20}}
	\end{pmatrix}\\$$$$\\
	
		$$q2' =\begin{pmatrix}
		 1\\
		 2\\
    	\end{pmatrix}-\begin{pmatrix}
		 \frac{10}{\sqrt{20}}
	\end{pmatrix}\begin{pmatrix}
		 \frac{2}{\sqrt{20}}\\
		  \frac{4}{\sqrt{20}}
	\end{pmatrix}\\$$$$\\
	
	$$q2' =\begin{pmatrix}
		 1\\
		 2\\
    	\end{pmatrix}-\begin{pmatrix}
		 \frac{20}{20}\\
		 \\
		 \frac{40}{20}\\
	\end{pmatrix}\\$$$$\\
	
	
	
	$$q2' =\begin{pmatrix}
		 {0}\\
		 {0}\\
	\end{pmatrix}\\$$$$\\
	
	
	Since,\\
		q2=\dfrac{q2'}{length of q2'}\\
	$$$$\\
	
	Therefore,\\
	
		$$q2=
	    \frac{1}{length of q2'}
	\begin{pmatrix}
		 0\\
		 0\\
	\end{pmatrix}\\
	
	
		Here,\\
		
	length of q2'=    $\sqrt{(0)^2+(0)^2}\\
		
		$length of q2'= 0
		
		\\
	
		
		
		Hence,Using\\
		q2=\dfrac{q2'}{length of q2'}\\
	$$$$\\
	we get,\\
		
		q2=0.\begin{pmatrix}
		 0\\
		 0\\
	\end{pmatrix}\\
		
		
		
		So,
		
			q2=\begin{pmatrix}
		 0\\
		 0\\
	\end{pmatrix}\\
		

	\newpage
		
		Now,\\
		Q can be obtained by combining 
		
		
		\\q1
		
		and\\
		
		\\q2
		
		Therefore,\\
		
		
	Q=\begin{pmatrix}
		\dfrac{2}{\sqrt{20}} &
		0\\
		\\
		 \dfrac{4}{\sqrt{20}}& 
		 0\\
	\end{pmatrix}
		\end{gather}
		\\
		
		Since we know\\
		
		A=QR\\
		
		Pre-Multiplying both sides by Q^T,\\
		
		we get,\\
		
		Q^TA=Q^TQR\\
		
		But Q^TQ=I \hspace{4cm}\textbf{[I means identity matrix]}\\
		
		So,\\
		Q^TA=IR\\
		Q^TA=R   \\
		
		
		Or,\\
		
		R=Q^TA\\
		
		\textbf{Hence,}\\
		R=  \begin{pmatrix}
		\dfrac{2}{\sqrt{20}} &
		 \dfrac{4}{\sqrt{20}}\\ 
		 0&
		 0\\
	\end{pmatrix}\begin{pmatrix}
		 2 & 1\\
		 4 & 2
    	\end{pmatrix}}
		\end{gather} \\
		
		\\
		R=  \begin{pmatrix}
		\dfrac{20}{\sqrt{20}} &
		 \dfrac{10}{\sqrt{20}}\\ 
		 0&
		 0\\
	\end{pmatrix}
		\\
		
	
		
		\\
	
	
	Therefore,\\
	A=QR\\
	
		\begin{pmatrix}
		 2 & 1\\
		 4 & 2\\
    	\end{pmatrix}=\begin{pmatrix}
		\dfrac{2}{\sqrt{20}}&
		0\\
	\dfrac{4}{\sqrt{20}}	&
		0\\
	\end{pmatrix}\begin{pmatrix}
		\dfrac{20}{\sqrt{20}} &
		 \dfrac{10}{\sqrt{20}}\\ 
		 0&
		 0\\
	\end{pmatrix}
    	\\$$$$\\
    	
	
	
	
	
	
	
	
	
	
	
	
	
	
	
\vfill\emph{The End}
	\end{document}